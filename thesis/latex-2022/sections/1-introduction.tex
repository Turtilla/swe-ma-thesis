\section{Introduction}
\label{sec:intro}

In his iconic quote, referenced in the title of this thesis, the famous Polish poet Mikołaj Rej declared his conviction that Polish people have their own language: 

\begin{quote}
\centering
   \textit{A niechaj narodowie wżdy postronni znają, \\
    Iż P o l a c y nie gęsi, iż swój język mają!}\footnote{ `And may the other nations finally know that Poles are not geese, that they have their own language!'} \\
\raggedleft
\citet{mikołaj_rej}
\end{quote}

While one would be hard-pressed to find someone trying to dispute the existence of a Polish language, one perhaps should consider certain alterations to the quote itself: the average Polish user no longer speaks in the same fashion as Mikołaj Rej did when he wrote his poetry in the 16\textsuperscript{th} century, and the variation occurs not only diachronically, but also regionally. Perhaps it would be more correct to say then that Poles \textit{swe języki mają!}\footnote{ `have their own languages!'}

Language variation plays an essential role in natural language processing: natural language as used by speakers is ever-changing, and NLP tools have to, to some extent, account for even the synchronic variation. As \citet{Zampieri2020NaturalLP} point out, developing methods to handle language variation is also relevant for adapting the existing tools to minority languages or dialects. While there are methods that allow for the utilization of raw data, \citet{ponti_2019} remarks that other methods still rely on annotated data which is difficult to come by for less popular languages. Simultaneously, as \citet{quantitative-historical} note, historical linguistics on its own is a data-driven field, and access to data, as well as methods for processing it, are very important; simultaneously, they highlight the usefulness of annotated corpora. This annotation may be especially useful when it comes to languages with a richer morphology, such as Slavic languages, as it may enable e.g. searching for all the inflectional forms of a given word \citep{pęzik_2012}. Thus, identifying the kinds of variation that occur between two languages or dialects not only constitutes a contribution to the body of knowledge about those languages on its own but also opens up possibilities for adapting existing tools for major languages to be used for the partial automatizing of data annotation. 

\subsection{Research Questions}
\label{subsec:research-questions}

The aims of this thesis can be described as two sides of the same coin, as it seeks to simultaneously answer the following two questions:
\begin{enumerate}
    \item \textit{Is it possible to identify language variation in terms of orthography, morphology, and syntax in a Polish text using tools such as lemmatizers, POS-taggers, and modern corpora?}
    \item \textit{In what ways does the text in question, a 19\textsuperscript{th}-century memoir by Juliusz Czermiński, differ from modern Polish?}
\end{enumerate}

\subsection{Motivation}
\label{subsec:motivation}

As mentioned at the start of this section, investigating language variation is not only its own field of linguistics but is also relevant for NLP - and methods and discoveries within these fields can inform each other. It is interesting to see how tools intended for working with modern languages can be used to identify ways in which those differ from their historical counterparts. These differences can help inform the pre-processing of the texts or ways in which the tools need to be adapted to enable a more reliable data annotation, which, in turn, can be used for further qualitative, corpus-based inquiries into some historical form of a language. 

Simultaneously, language variation is not only a thing of the past, but a continuous process that may lead to modern tools becoming outdated in the future; furthermore, language varies also based on factors other than time, such as geography or social class, and this variation can also prove problematic to NLP applications. While methods such as cross-linguistic transfer learning do exist, they cannot be applied to all the tools equally, especially non-neural ones or ones that are not available for fine-tuning.

While the major focus of this thesis is on exploring historical language variation in Polish alongside the methods that can be employed for such investigations, hopefully, it can yield insights into and spark some discussion about related topics, such as the handling of linguistic variation in NLP, computational methods in historical linguistics, as well as resources for historical linguistics and the annotation thereof.

\subsection{Contributions}
\label{subsec:contributions}

Within this thesis, a variety of ways for discovering historical linguistic variation using tools intended for modern languages is tested. The methods are reviewed with regard to the kind of results that they yield and the amount of annotation or preparation needed. While the majority of the methods do require the data to be annotated, some quite interesting observations can be made based on the results. Simultaneously, due to the nature of the experiments, the thesis unwittingly conducts a comparison of the performance of multiple tools (part-of-speech taggers and lemmatizers) on modern data, which can inform the choice of a tool for other research or real-life application. The manually annotated data that this project is centered around constitutes a non-negligible contribution to the body of annotated historical data for Polish. Finally, certain issues raised throughout the thesis can hopefully contribute to the discussion concerning the tagsets used for annotating data in Polish and certain biases present in the data.

\subsection{Scope}
\label{subsec:scope}

It is important to note that the data used in this project is not very big, and it does originate from a single author. Therefore, instead of trying to make claims about the language of the time and region, the focus is laid on the author's particular idiolect. At the same time, there likely are many more methods that could be used, directly or indirectly, for identifying language variation. In addition, some levels at which language can differ (phonology, pragmatics) are omitted entirely, and some are not discussed as in-depth as others. Instead, the project focuses on the aspects detailed in \autoref{subsec:research-questions}.