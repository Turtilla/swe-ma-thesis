\section{Ethical Considerations}
\label{sec:ethicalcons}

Ethical concerns are ever-present within the field of Natural Language Processing. As \citet{hovy-spruit-2016-social} point out, these concerns can revolve both around the data itself and the impact that NLP can have on the society. \citet{stochastic-parrots} point out the environmental impact of computationally-heavy processes (such as training very large language models) and draw the readers' attention to how biases existing in the training data can impact the aforementioned models. A number of different tools and resources were utilized in this thesis, and many of them deserve to be discussed from an ethical point of view.

To begin with, the experiments conducted as a part of this thesis did not involve training of any large models from scratch - and the most computationally expensive part was the fine-tuning of two BERT-based part-of-speech taggers. While training a large transformer model like BERT is definitely impactful, its ability to be fine-tuned for different applications eliminates the need to train another costly model from scratch. Utilizing pre-existing, optimized code for token tagging suited for this model likely streamlined the process as well. With the exception of Marmot, the training of which does not appear to be computationally expensive, the other taggers were already pre-trained, which made this investigation much more justifiable than training many models from the start would be with the environmental impact in mind. 

As for the models and their biases it is important to point out that any biases that may be noticeable in the tools' outputs are most likely a reflection of the biases in the training data. The author of the Polish version of BERT, \citet{kłeczek_2021}, points out himself that the data used to pre-train his models may include such biases. As \citet{wroblewska-2018-extended} explains, while the Polish Dependency Bank is largely based on the National Corpus of Polish, it does include sentences from other sources. Not much information is provided concerning how balanced and representative PDB is, so while an assumption is made within this thesis that it is representative of modern Polish, it need not be so. Although much care was put into making the National Corpus of Polish representative of the whole language community and balanced in terms of the featured sources, it is impossible to assert that the included texts do not reflect biases related to gender, sexuality, ethnicity, etc., and the same is true for the PDB \citep{nkjp}. An interesting issue related to the annotation of the PDB was raised in the \autoref{subsec:xpos-tagging}, as first- and second-person singular pronouns are annotated for gender despite not overtly displaying it. During the annotation, the taggers did annotate pronouns used by a female speaker as masculine, potentially reflecting a gender bias in the training data (the PDB training set). 

While both the out-of-context sentences provided in PDB and the limited access that users have to the texts that constitute the National Corpus of Polish are methods for dealing with copyright and privacy issues, working with an independently transcribed and annotated text may pose its own ethical problems. However, in the case of this thesis the data in question is historical, and its author has passed away around a century ago, which largely voids the issue of the author's consent for the use of his text. Nevertheless, the issue of potential biases and opinions featured in the text, including potentially offensive words, remains and needs to be taken into consideration.

Another issue worth considering in the light of this thesis is the representation of small languages or dialects and the regional and diachronic variation in NLP. \citet{mcenery-etal-2000-corpus} and \citet{soria-etal-2016-fostering} point out that the lack of corpus data for such languages severely impedes the development of appropriate NLP tools, which can, in turn, lead to some social groups being excluded from utilizing such tools or result in their language becoming more endangered. While the data analyzed in this thesis is not a sample of a modern minority language or dialect, some of the methods tested in the experiments could be used for exploring contemporary language variation as well, potentially contributing to solving this issue. 



