\section{Conclusions}
\label{sec:conclusions}

The aim of this thesis was to explore some of the potential methods for identifying language variation in Polish on the example of a late 19\textsuperscript{th}-century memoir by Juliusz Czermiński and the ways the language of the memoir differs from modern Polish. The text was expected to differ from modern standard Polish in some ways due to its age and geographical origin. A part of the text was manually annotated with lemmas, UPOS, and XPOS tags according to the UD standards for such annotation. Subsequently, a number of experiments were conducted, where the memoir was compared to the test set of PDB-UD, the largest existing UD treebank for modern Polish. The experiments included comparing the performance of various tagging and lemmatization tools on the two sets of data, reviewing the features of the most problematic tokens, part-of-speech tag statistics analyses, and a review of which tokens and lemmas from the data are not present in a specific subsection of the National Corpus of Polish.

The results, presented and discussed in \autoref{sec:results}, show that the memoir does differ from resources that are available for modern Polish, and some major trends in terms of spelling variation (the use of \textit{y} for the /j/ phoneme, the use of \textit{e} where modern Polish features the phoneme /a/, spelling the negation of a verb together with the verb itself) are identified. A noticeable drop in performance can be observed for all three categories of annotation tools, regardless of their architecture (although some perform better than others). N-gram counts of the part-of-speech tags suggest possible word order or syntactic differences but are inconclusive. A comparison of the memoir's vocabulary reveals a number of tokens that are not present in the National Corpus of Polish in the selected timespan; while some of those are proper names, other examples show spelling and vocabulary variation. Inevitably, the methods explored in this project have their drawbacks, the major one being that most of them require the data to be annotated in some way. Simultaneously, this thesis offers a small contribution to the body of annotated historical data for Polish and advocates for the usefulness of constructing larger collections of diachronic data with annotation compatible with other large annotated corpora. One more side effect of the tagger and lemmatizer experiments is that they provide a comparison of the performance of various tools on modern data.

The experiments and the results presented in this thesis explore the ways in which existing tools for modern languages can help identify language variation in historical texts. While the presented solutions may not be perfect, they encourage further discussion and research into utilizing them for diachronic linguistics, not only for simply identifying the language variation but also as an intermediate step in the process of automatizing the annotation of historical data for the creation of larger corpora. 