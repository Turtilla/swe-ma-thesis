\section{Resources}
\label{app-resources}

Within this Appendix, links to a number of resources, both created and used in the thesis, will be provided, alongside explanations; these resources have been cited in the text according to the resource-specific guidelines (if provided):
\begin{itemize}
    \item Thesis repository, including the text of the memoir with and without various annotation (the repository structure is explained in the README): \url{https://github.com/Turtilla/swe-ma-thesis}
    \item Instructions and contact information for the PELCRA search engine for the National Corpus of Polish\footnote{If one desires the so-called programming access, one must contact the person listed on this page, as this access is limited.}: \url{http://www.nkjp.uni.lodz.pl/help.jsp}
    \item The University of Sheffield UD-based POS tagger\footnote{As of 26.04.2023 this tagger appears to not work due to an "Internal Server Error".}: \url{https://cloud.gate.ac.uk/shopfront/displayItem/tagger-pos-pl-maxent1}
    \item Morfeusz2: \url{http://morfeusz.sgjp.pl/}
    \item Concraft-pl: \url{https://github.com/kawu/concraft-pl}
    \item Transformers' Token Classification: \\ \url{https://github.com/huggingface/transformers/tree/main/examples/legacy/token-classification}
    \item BERT for Polish (cased): \\ \url{https://huggingface.co/dkleczek/bert-base-polish-cased-v1}
    \item Marmot: \url{http://cistern.cis.lmu.de/marmot/}
    \item Stanza: \url{https://stanfordnlp.github.io/stanza/}
\end{itemize}
\newpage

\section{Error Type Definitions}
\label{error-types}

\renewcommand{\arraystretch}{1.25}
\begin{table}[H]
\begin{center}
\scalebox{1}{
\begin{tabular}{lll}
\toprule \bf Error Type & \bf Definition & \bf Included Subtypes \\ \toprule

spelling & \makecell[l]{Any spelling-related differences, \\ both intentional and not} & \textit{y}, \textit{nie}, spelling, capitalization, \textit{e} \\

name & \makecell[l]{Any type of proper names} & proper name, surname, name \\

abbreviation & Abbreviated tokens & abbreviation \\

ambiguous & \makecell[l]{The whole token or its part \\ is ambiguous in some way} & ambiguous, problematic \\

unidentified & \makecell[l]{The reason for the error \\ cannot be identified} & unidentified \\

vocabulary & \makecell[l]{The token is likely OOV \\ due to being specialized, \\ dialectical, archaic, or foreign} & foreign \\

grammar & \makecell[l]{The token displays an unusual \\ grammatical feature} & grammar \\
\bottomrule
\end{tabular}}
\end{center}
\caption{\label{table:general-error-type-explanations} General types of errors made by lemmatizers.}
\end{table}
\newpage

\renewcommand{\arraystretch}{1.25}
\begin{table}[H]
\begin{center}
\scalebox{1}{
\begin{tabular}{p{2cm}p{4.5cm}p{3.5cm}p{1.75cm}p{1.75cm}}
\toprule \bf Error Type & \bf Definition & \bf Example & \bf Predictions & \bf Standard\\ \toprule
spelling: \textit{y} & \makecell[l]{The grapheme \textit{y} is used \\ instead of \textit{j} to signify \\ the /j/ sound} & \makecell[l]{\textit{suchey} \\ `dry'} & suchey & suchy \\ 

\makecell[l]{name: other} & \makecell[l]{Potentially unfamiliar \\ proper name token} & \makecell[l]{\textit{Bludniki} \\ `Bludniki'} & \makecell[l]{Bludnik \\ bludnik} & Bludniki  \\ 

spelling: \textit{nie} & \makecell[l]{Spelling of the negation \\ with the negated word \\ in word classes that \\ normally do not allow it} & \makecell[l]{\textit{niemają} \\ `(they) don't have'} & \makecell[l]{niemaja \\ nie} & niemieć  \\ 

\makecell[l]{spelling: other} & \makecell[l]{Other spelling differences} & \makecell[l]{\textit{ładąn} \\ `pretty'} & ładąn & ładna \\ 

\makecell[l]{name: \\ surname} & \makecell[l]{Potentially unfamiliar \\ surname token}  & \makecell[l]{\textit{Polanowski} \\ `Polanowski'} & polanowski & Polanowski \\ 

\makecell[l]{spelling: \\ capitalization} & \makecell[l]{Nonstandard capitalization} & \makecell[l]{\textit{Dziedzica} \\ `of the heir'} & \makecell[l]{Dziedzic \\ dziedzica} & dziedzic  \\ 

abbreviation & \makecell[l]{The token is abbreviated} & \makecell[l]{\textit{Stan} \\ `Stan'} & \makecell[l]{Stan \\ stan} & Stanisław \\

spelling: \textit{e} & \makecell[l]{The grapheme \textit{e} is used \\ instead of another vowel \\ (commonly \textit{y})} & \makecell[l]{\textit{tem} \\ `this'} & \makecell[l]{tema \\ tem} & to \\

\makecell[l]{ambiguous: \\ other} & \makecell[l]{The token could have \\ more than one interpretation} & \makecell[l]{\textit{dobra} \\`goods'} & dobry & dobra \\ 

\makecell[l]{name: \\ given name} & \makecell[l]{A potentially unfamiliar \\ first name token} & \makecell[l]{\textit{Kleosię} \\ `Kleosia'} & \makecell[l]{Kleosię \\ kleosia} & Kleosia \\

unidentified & \makecell[l]{No apparent reason} & \makecell[l]{\textit{łania} \\ `doe'} & \makecell[l]{łani \\ łanie} & łania \\

\makecell[l]{ambiguous: \\ problematic} & \makecell[l]{The choice of the lemma \\ is up to the annotator \\ because of two acceptable \\ spelling variants or the \\ word being on the verge \\ of being independent or \\ being a derivational form \\ of another word} & \makecell[l]{\textit{bombardowaniu} \\ `of the bombing'} & \makecell[l]{bombar-\\dować} & \makecell[l]{bombardo-\\wanie} \\

\makecell[l]{vocabulary: \\ foreign} & The token is foreign & \makecell[l]{\textit{Toje} \\ `-'} & \makecell[l]{Toje \\ tój} & toje \\

\makecell[l]{grammar: \\ other} & \makecell[l]{The token displays an \\ unusual grammatical ending} & \makecell[l]{\textit{człowiecze} \\ `human'} & \makecell[l]{człowieczy \\ człowiec} & człowiek \\

\bottomrule
\end{tabular}}
\end{center}
\caption{\label{table:error-type-explanations} Types and examples of errors made by lemmatizers. The translations into English are not ideal since they do not capture all of the encoded information, such as case, gender, number.}
\end{table}
\newpage

\renewcommand{\arraystretch}{1.25}
\begin{table}[H]
\begin{center}
\scalebox{1}{
\begin{tabular}{lll}
\toprule \bf Error Type & \bf Definition & \bf Included Subtypes \\ \toprule

spelling & \makecell[l]{Any spelling-related differences, \\ both intentional and not} & capitalization, \textit{y}, \textit{e}, \textit{nie}, spelling \\

name & \makecell[l]{Any type of proper names} & surname, proper name, name \\

abbreviation & Abbreviated tokens & abbreviation \\

ambiguous & \makecell[l]{The whole token or its part \\ is ambiguous in some way} & ambiguous, UD, ending, problematic, digits \\

unidentified & \makecell[l]{The reason for the error \\ cannot be identified} & unidentified \\

vocabulary & \makecell[l]{The token is likely OOV \\ due to being specialized, \\ dialectical, archaic, or foreign} & archaic, foreign, uncommon, special \\

grammar & \makecell[l]{The token displays an unusual \\ grammatical feature} & impersonal, grammar \\

\bottomrule
\end{tabular}}
\end{center}
\caption{\label{table:general-upos-error-type-explanations} General types of errors made by UPOS taggers.}
\end{table}
\newpage

\renewcommand{\arraystretch}{1.5}
\begin{longtable}[H]{p{2cm}p{4.5cm}p{3.5cm}p{1.75cm}p{1.75cm}}
%\begin{center}
%\begin{tabular}{p{2cm}p{4.5cm}p{3.5cm}p{1.75cm}p{1.75cm}}
\toprule \bf Error Type & \bf Definition & \bf Example & \bf Predictions & \bf Standard\\ \toprule

\makecell[l]{ambiguous: \\ other} & \makecell[l]{The token is ambiguous} & \makecell[l]{\textit{jego} \\ `his'} & PRON & DET \\ 

\makecell[l]{spelling: \\ capitalization} & Nonstandard capitalization & \makecell[l]{\textit{Patrona} \\ `patron'} & \makecell[l]{PROPN \\ NOUN } & NOUN  \\ 

spelling: \textit{y} & \makecell[l]{The grapheme \textit{y} is used \\ instead of \textit{j} to signify \\ the /j/ sound} & \makecell[l]{\textit{móy} \\ `my'} & \makecell[l]{\\ PROPN \\ ADJ \\ VERB} & DET \\ 

unidentified & \makecell[l]{No apparent reason} & \makecell[l]{\textit{wyłącznie} \\ `exclusively'} & \makecell[l]{\\ PART \\ ADV \\ NOUN} & ADV \\ 

\makecell[l]{vocabulary: \\ archaic} & \makecell[l]{The token is somewhat \\ archaic or regional} & \makecell[l]{\textit{pono} \\ `supposedly'} & \makecell[l]{\\ PUNCT \\ VERB \\ PROPN \\ ADJ} & PART \\ 

\makecell[l]{ambiguous: \\ UD} & \makecell[l]{The token has more \\ than one possible tag \\ due to UD guidelines} & \makecell[l]{\textit{był} \\ `(there) was'} & \makecell[l]{VERB \\ AUX} & VERB \\ 

\makecell[l]{name: \\ surname} & \makecell[l]{Potentially unfamiliar \\ surname token}  & \makecell[l]{\textit{Ostaszewskiej} \\ `Ostaszewska'} & \makecell[l]{ADJ \\ PROPN} & PROPN \\ 

spelling: \textit{e} & \makecell[l]{The grapheme \textit{e} is used \\ instead of another vowel \\ (commonly \textit{y})} & \makecell[l]{\textit{małem} \\ `small'} & \makecell[l]{ADJ \\ NOUN} & ADJ \\ 

spelling: \textit{nie} & \makecell[l]{Spelling of the negation \\ with the negated word \\ in word classes that \\ normally do not allow it} & \makecell[l]{\textit{niechciały} \\ `(they) didn't want to'} & \makecell[l]{VERB \\ NOUN \\ ADJ} & VERB  \\ 

\makecell[l]{spelling: \\ other} & \makecell[l]{Other spelling differences} & \makecell[l]{\textit{wkońcu} \\ `in the end'} & \makecell[l]{ADV \\ NOUN} & NOUN \\ 

\makecell[l]{ambiguous: \\ ending} & \makecell[l]{The ending of the word \\ can be indicative of \\ more than one class} & \makecell[l]{\textit{chwała} \\ `glory'} & \makecell[l]{NOUN \\ VERB} & NOUN \\ 

\makecell[l]{name: \\ other} & \makecell[l]{Potentially unfamiliar \\ proper name token} & \makecell[l]{\textit{Dąbrowy} \\ `Dąbrowa'} & \makecell[l]{PROPN \\ ADJ} & PROPN  \\ 

\makecell[l]{ambiguous: \\ problematic} & \makecell[l]{The choice of the tag \\ is up to the annotator \\ because of two spelling \\ variants or the word \\ having been derived} & \makecell[l]{\textit{służąca} \\ `servant'} & \makecell[l]{NOUN \\ ADJ} & \makecell[l]{NOUN} \\ 

\makecell[l]{ambiguous: \\ digits} & \makecell[l]{The token is in digits} & \makecell[l]{\textit{1} \\ `1'} & \makecell[l]{ADJ \\ NUM} & NUM \\ 

\makecell[l]{vocabulary: \\ foreign} & The token is foreign & \makecell[l]{\textit{daruju} \\ `-'} & \makecell[l]{\\ NOUN \\ VERB} & X \\ 

\makecell[l]{vocabulary: \\ uncommon} & The token is uncommon & \makecell[l]{\textit{czółno} \\ `canoe'} & \makecell[l]{ADV \\ ADJ} & NOUN \\ 

abbreviation & \makecell[l]{The token is abbreviated} & \makecell[l]{\textit{5-cioro} \\ `five'} & \makecell[l]{\\ NUM \\ NOUN \\ MIS-\\PARSED} & NUM \\ 

\makecell[l]{grammar: \\ impersonal} & \makecell[l]{The token is an \\ impersonal verb form} & \makecell[l]{\textit{wierzono} \\ `(it was) believed'} & \makecell[l]{\\ VERB \\ ADJ \\ ADV} & VERB \\ 

\makecell[l]{name: \\ given name} & \makecell[l]{A potentially unfamiliar \\ first name token} & \makecell[l]{\textit{Wiktorów} \\ `Wiktors'} & \makecell[l]{\\ PROPN \\ NOUN} & PROPN \\

\makecell[l]{grammar: \\ other} & \makecell[l]{The token features a \\ nonstandard inflectional \\ ending} & \makecell[l]{\textit{egzamina} \\ `exams'} & \makecell[l]{NOUN \\ ADJ} & NOUN \\

\makecell[l]{vocabulary: \\ stylized} & \makecell[l]{The token is a intentionally \\ spelled in a nonstandard \\ fashion} & \makecell[l]{\textit{psipiólki} \\ `quails'} & \makecell[l]{NOUN \\ ADJ} & NOUN \\ 

\bottomrule
%\end{tabular}
%\end{center}
\caption{\label{table:error-type-upos-explanations} Types and examples of errors made by the UPOS taggers. The translations into English are not ideal since they do not capture all of the encoded information, such as case, gender, number.}
\end{longtable}
\newpage

\renewcommand{\arraystretch}{1.25}
\begin{table}[H]
\begin{center}
\scalebox{1}{
\begin{tabular}{lll}
\toprule \bf Error Type & \bf Definition & \bf Included Subtypes \\ \toprule

spelling & \makecell[l]{Any spelling-related differences, \\ both intentional and not} &  \textit{y}, \textit{nie}, spelling, \textit{e} \\

name & \makecell[l]{Any type of proper names} & proper name, surname, name \\

abbreviation & Abbreviated tokens & abbreviation \\

ambiguous & \makecell[l]{The whole token or its part \\ is ambiguous in some way} & ambiguous, digits, problematic, currency, \\

unidentified & \makecell[l]{The reason for the error \\ cannot be identified} & unidentified \\

vocabulary & \makecell[l]{The token is likely OOV \\ due to being specialized, \\ dialectical, archaic, or foreign} & archaic, foreign, uncommon \\

grammar & \makecell[l]{The token displays an unusual \\ grammatical feature} & gender, vocative \\

\bottomrule
\end{tabular}}
\end{center}
\caption{\label{table:general-uxpos-error-type-explanations} General types of errors made by XPOS taggers.}
\end{table}
\newpage

\renewcommand{\arraystretch}{1.5}
\begin{longtable}[H]{p{2cm}p{4cm}p{1.5cm}p{3cm}p{3cm}}
%\begin{center}
%\begin{tabular}{p{2cm}p{4.5cm}p{3.5cm}p{1.75cm}p{1.75cm}}
\toprule \bf Error Type & \bf Definition & \bf Example & \bf Predictions & \bf Standard\\ \toprule

\makecell[l]{ambiguous: \\ other} & \makecell[l]{The token's meaning is \\ ambiguous} & \makecell[l]{\textit{parafii} \\ `parish'} & subst:sg:gen:f & subst:sg:loc:f \\ 

unidentified & \makecell[l]{No apparent reason} & \makecell[l]{\textit{Ciotka} \\ `aunt'} & \makecell[l]{\\ subst:sg:acc:m1 \\ subst:sg:nom:m1 \\ subst:sg:nom:f} & subst:sg:nom:f \\

\makecell[l]{name: \\ other} & \makecell[l]{Potentially unfamiliar \\ proper name token} & \makecell[l]{\textit{Brzeżan} \\ `Brzeżany'} & \makecell[l]{\\subst:pl:gen:n:pt \\ subst:sg:gen:n:ncol \\ subst:pl:gen:m1 \\ subst:sg:gen:m1} & subst:pl:gen:n:pt  \\ 

spelling: \textit{y} & \makecell[l]{The grapheme \textit{y} is used \\ instead of \textit{j} to signify \\ the /j/ sound} & \makecell[l]{\textit{arye} \\ `arias'} & \makecell[l]{subst:pl:acc:m3 \\ adj:pl:nom:m2:pos} & subst:pl:acc:f \\ 

\makecell[l]{ambiguous: \\ digits} & \makecell[l]{The token is in digits} & \makecell[l]{\textit{1808} \\ `1808'} & \makecell[l]{adj:sg:gen:m3:pos \\ dig} & adj:sg:gen:m3:pos \\

\makecell[l]{ambiguous: \\ problematic} & \makecell[l]{The choice of the tag \\ is up to the annotator \\ because of two spelling \\ variants or the word \\ having been derived} & \makecell[l]{\textit{oczytany} \\ `learned'} & \makecell[l]{adj:sg:nom:m1:pos\\ adj:sg:nom:m3:pos \\ ppas:sg:nom:\\\hspace{0.5cm}m1:perf:aff} & \makecell[l]{ppas:sg:nom:\\\hspace{0.5cm}m1:perf:aff} \\ 

spelling: \textit{nie} & \makecell[l]{Spelling of the negation \\ with the negated word \\ in word classes that \\ normally do not allow it} & \makecell[l]{\textit{niema} \\ `(there) \\ aren't'} & \makecell[l]{fin:sg:ter:imperf \\ subst:sg:nom:f} & fin:sg:ter:imperf  \\ 

\makecell[l]{spelling: other} & \makecell[l]{Other spelling differences} & \makecell[l]{\textit{kończ} \\ `end'} & \makecell[l]{subst:pl:gen:n:ncol \\ subst:pl:gen:f \\ impt:sg:sec:imperf} & subst:sg:gen:m3 \\ 

\makecell[l]{vocabulary: \\ archaic} & \makecell[l]{The token is somewhat \\ archaic or regional} & \makecell[l]{\textit{wieleż} \\ `many'} & \makecell[l]{num:pl:acc:m3:rec \\ subst:sg:nom:m3 \\ subst:sg:nom:m1} & num:pl:acc:m3:rec \\ 

\makecell[l]{vocabulary: \\ foreign} & The token is foreign & \makecell[l]{\textit{Toje} \\ `-'} & \makecell[l]{subst:sg:nom:n:ncol \\ xxx \\ subst:sg:nom:m1} & ign \\ 

\makecell[l]{name: \\ surname} & \makecell[l]{Potentially unfamiliar \\ surname token}  & \makecell[l]{\textit{Zabilskich} \\ `Zabilscy'} & \makecell[l]{subst:pl:gen:m1 \\ adj:pl:gen:f:pos \\ subst:pl:acc:m1} & subst:pl:gen:m1 \\ 

\makecell[l]{vocabulary: \\ uncommon} & The token is uncommon & \makecell[l]{\textit{hoża} \\ `swift'} & \makecell[l]{adj:sg:nom:f:pos \\ subst:sg:nom:f} & adj:sg:nom:f:pos \\ 

\makecell[l]{ambiguous: \\ currency} & \makecell[l]{The token is a name \\ of a currency and \\ belongs to the \textit{m2} gender} & \makecell[l]{\textit{dukatów} \\ ducats} & \makecell[l]{subst:pl:gen:m3 \\ subst:pl:gen:m2} & subst:pl:gen:m2 \\

spelling: \textit{e} & \makecell[l]{The grapheme \textit{e} is used \\ instead of another vowel \\ (commonly \textit{y})} & \makecell[l]{\textit{któremi} \\ `(with)\\which'} & \makecell[l]{adj:pl:inst:n:pos \\ subst:pl:inst:m3 \\ subst:pl:inst:m1 \\ adj:pl:inst:f:pos} & adj:pl:inst:m1:pos \\ 

\makecell[l]{grammar: \\ gender} & \makecell[l]{The token is assigned \\ the wrong gender} & \makecell[l]{\textit{Ja} \\ `I'} & \makecell[l]{\\ppron12:sg:nom:\\\hspace{0.5cm}m1:pri \\ {}} & \makecell[l]{ppron12:sg:nom:\\\hspace{0.5cm}f:pri} \\

\makecell[l]{grammar: \\ vocative} & \makecell[l]{The vocative case is \\ not properly recognized} & \makecell[l]{\textit{Ty} \\ `you'} & \makecell[l]{ppron12:sg:nom:\\\hspace{0.5cm}m1:sec} & \makecell[l]{ppron12:sg:voc:\\\hspace{0.5cm}m1:sec} \\

abbreviation & \makecell[l]{The token is abbreviated} & \makecell[l]{\textit{śp} \\ `may his \\ soul rest \\ in peace'} & \makecell[l]{ \\ brev:pun \\ subst:sg:nom:m3 \\ subst:sg:nom:m1 \\ aglt:sg:sec:\\\hspace{0.5cm}imperf:nwok \\ {}} & brev:npun \\ 

\makecell[l]{name: \\ given name} & \makecell[l]{A potentially unfamiliar \\ first name token} & \makecell[l]{\textit{Melchior} \\ `Melchior'} & \makecell[l]{subst:sg:nom:m1 \\ subst:sg:nom:m3 \\ subst:sg:gen:f} & subst:sg:nom:m1 \\ 

\bottomrule
%\end{tabular}
%\end{center}
\caption{\label{table:error-type-xpos-explanations} Types and examples of errors made by the XPOS taggers. The translations into English are not ideal since they do not capture all of the encoded information, such as case, gender, number.}
\end{longtable}
\newpage

\section{Extended UPOS measures}
\label{upos-class-measures}

\begin{table}[H]
\begin{center}
\scalebox{1.1}{
\begin{tabular}{|l|cc|cc|}
\hline \bf POS-tag & \multicolumn{2}{c|}{\bf Modern} & \multicolumn{2}{c|}{\bf Historical} \\
\bf {} & \bf Precision & \bf Recall & \bf Precision & \bf Recall \\ \hline
ADJ & 99.11\% & 99.55\% & 94.23\% & 93.31\% \\
ADP & 99.77\% & 99.91\% & 99.74\% & 98.74\% \\
ADV & 97.73\% & 98.44\% & 87.61\% & 89.94\% \\
AUX & 98.93\% & 98.93\% & 91.32\% & 84.03\% \\
CCONJ & 97.64\% & 97.99\% & 98.84\% & 93.92\%\\
DET & 98.82\% & 99.06\% & 94.52\% & 83.99\% \\
INTJ & 87.50\% & 70.00\% & 0.00\% & 0.00\% \\
NOUN & 99.58\% & 99.26\% & 95.47\% & 95.27\% \\
NUM & 97.06\% & 99.62\% & 98.21\% & 82.71\% \\
PART & 97.14\% & 95.12\% & 79.45\% & 84.47\% \\
PRON & 99.62\% & 99.44\% & 93.69\% & 91.09\% \\
PROPN & 95.88\% & 98.12\% & 84.07\% & 96.87\% \\
PUNCT & 99.96\% & 99.98\% & 99.59\% & 100.00\% \\
SCONJ & 98.68\% & 98.40\% & 87.91\% & 94.97\% \\
SYM & 50.00\% & 25.00\% & 0.00\% & 0.00\% \\
VERB & 99.72\% & 99.72\% & 95.42\% & 96.01\% \\
X & 95.62 \% & 91.91\% & 82.76\% & 72.73\% \\
\hline
\end{tabular}}
\end{center}
\caption{\label{table:bert-pr} BERT precision and recall per POS-tag per test set. }
\end{table}

\begin{table}[H]
\begin{center}
\scalebox{1.1}{
\begin{tabular}{|l|cc|cc|}
\hline \bf POS-tag & \multicolumn{2}{c|}{\bf Modern} & \multicolumn{2}{c|}{\bf Historical} \\
\bf {} & \bf Precision & \bf Recall & \bf Precision & \bf Recall \\ \hline
ADJ & 97.25\% & 97.71\% & 81.33\% & 84.57\% \\
ADP & 99.46\% & 99.74\% & 99.58\% & 98.99\%  \\
ADV & 95.59\% & 95.33\% & 86.83\% & 85.80\% \\
AUX & 91.67\% & 95.60\% & 85.02\% & 86.31\% \\
CCONJ & 96.17\% & 95.60\% & 97.20\% & 95.76\%\\
DET & 98.44\% & 96.93\% & 95.85\% & 74.94\% \\
INTJ & 46.15\% & 60.00\% & 0.00\% & 0.00\% \\
NOUN & 98.23\% & 98.04\% & 89.16\% & 91.16\% \\
NUM & 98.04\% & 94.34\% & 97.98\% & 72.93\% \\
PART & 93.49\% & 92.42\% & 78.39\% & 75.73\% \\
PRON & 99.05\% & 98.31\% & 90.66\% & 86.53\% \\
PROPN & 91.30\% & 94.09\% & 79.01\% & 86.20\% \\
PUNCT & 99.95\% & 99.95\% & 100.00\% & 100.00\% \\
SCONJ & 96.49\% & 96.21\% & 86.73\% & 91.96\% \\
SYM & 100.00\% & 25.00\% & 0.00\% & 0.00\% \\
VERB & 97.96\% & 97.43\% & 91.67\% & 92.73\% \\
X & 89.33\% & 86.73\% & 63.16\% & 54.55\% \\
\hline
\end{tabular}}
\end{center}
\caption{\label{table:marmot-pr} Marmot precision and recall per POS-tag per test set. }
\end{table}

\begin{table}[H]
\begin{center}
\scalebox{1.1}{
\begin{tabular}{|l|cc|cc|}
\hline \bf POS-tag & \multicolumn{2}{c|}{\bf Modern} & \multicolumn{2}{c|}{\bf Historical} \\
\bf {} & \bf Precision & \bf Recall & \bf Precision & \bf Recall \\ \hline
ADJ & 98.17\% & 98.99\% & 88.87\% & 93.85\% \\
ADP & 99.46\% & 99.91\% & 99.58\% & 99.07\%  \\
ADV & 94.58\% & 96.06\% & 91.16\% & 88.46\% \\
AUX & 95.44\% & 97.14\% & 84.70\% & 86.31\% \\
CCONJ & 95.47\% & 96.17\% & 98.14\% & 97.24\%\\
DET & 98.00\% & 98.47\% & 94.49\% & 79.58\% \\
INTJ & 100.00\% & 50.00\% & 0.00\% & 0.00\% \\
NOUN & 99.17\% & 98.70\% & 95.15\% & 93.44\% \\
NUM & 98.48\% & 98.11\% & 97.25\% & 79.70\% \\
PART & 95.01\% & 90.97\% & 93.33\% & 74.76\% \\
PRON & 98.63\% & 98.87\% & 90.08\% & 91.68\% \\
PROPN & 94.14\% & 96.51\% & 79.07\% & 91.89\% \\
PUNCT & 99.95\% & 99.95\% & 99.59\% & 100.00\% \\
SCONJ & 95.86\% & 94.61\% & 86.30\% & 94.97\% \\
SYM & 100.00\% & 25.00\% & 0.00\% & 0.00\% \\
VERB & 99.20\% & 98.66\% & 93.73\% & 94.15\% \\
X & 93.53\% & 93.53\% & 73.58\% & 59.09\% \\
\hline
\end{tabular}}
\end{center}
\caption{\label{table:stanza-pr} Stanza precision and recall per POS-tag per test set. }
\end{table}

%% rerun?
\begin{table}[H]
\begin{center}
\scalebox{1.1}{
\begin{tabular}{|l|cc|cc|}
\hline \bf POS-tag & \multicolumn{2}{c|}{\bf Modern} & \multicolumn{2}{c|}{\bf Historical} \\
\bf {} & \bf Precision & \bf Recall & \bf Precision & \bf Recall \\ \hline
ADJ & 83.86\% & 91.58\% & 66.73\% & 75.08\% \\ 
ADP & 96.65\% & 98.89\% & 96.18\% & 97.64\%  \\
ADV & 79.69\% & 75.89\% & 75.09\% & 64.20\% \\
AUX & 87.67\% & 82.98\% & 84.08\% & 78.33\% \\
CCONJ & 93.72\% & 87.14\% & 95.58\% & 95.58\%\\
DET & 94.64\% & 72.96\% & 94.58\% & 44.55\% \\
INTJ & 0.00\% & 0.00\% & 0.00\% & 0.00\% \\
NOUN & 90.82\% & 92.91\% & 80.08\% & 82.56\% \\
NUM & 73.62\% & 70.57\% & 77.00\% & 57.89\% \\
PART & 89.62\% & 80.69\% & 83.77\% & 62.62\% \\
PRON & 94.35\% & 94.06\% & 85.54\% & 81.98\% \\
PROPN & 83.40\% & 92.29\% & 69.77\% & 92.60\% \\
PUNCT & 99.93\% & 99.72\% & 100.00\% & 99.92\% \\
SCONJ & 88.62\% & 63.56\% & 83.81\% & 44.22\% \\
SYM & 0.00\% & 0.00\% & 0.00\% & 0.00\% \\
VERB & 89.46\% & 88.86\% & 81.64\% & 86.35\% \\
X & 52.24\% & 41.42\% & 56.60\% & 45.45\% \\
\hline
\end{tabular}}
\end{center}
\caption{\label{table:ud-pr} UD Cloud tagger precision and recall per POS-tag per test set. }
\end{table}
\newpage

\section{National Corpus of Polish vocabulary comparison output}
\label{app:nkjp}

Aside from counts and proportions of the tested vocabulary that could not be found in the National Corpus of Polish, specific tokens and lemmas were returned. The items listed below were not found in the selected subsection of the National Corpus of Polish. They are separated by whitespace.

\begin{itemize}
    \item \textbf{PDB lemmas:} ! ) 19:15 25-procentowy 642-65-85 9-miesięczny Arasyb Bielsko-Biała Bushill-Matthews Collridge Eija-Riitta HA-Il Hawełko III-1 IRSC IRSR Instagram Lunzie McMillan-Scott Minecraft PPE-DE Palmiak Stallarholmen Winfryd-Bonifacy Yeosol [ ajtemik antysubsydyjny bezfabularny bio-obrazowanie ciemku krio-elektronowy lajwik merozoit niekonscjentywny non-profit nudności odmaterializować podwaliny przeciwbiałaczkowy przeciwretrowirusowy tekstilandia trichlorobenzen Ździara
    \item \textbf{PDB tokens:} ! " \% ' '' ( ) , 19:15 25-procentowy 5-proc 642-65-85 9-miesięczna ; Ajtemików Bushill-Matthewsowi Collridge Eija-Riitta HA-Il III-1 IRSC IRSR Instagramie Kalkilli Lunzie Maggego McMillan-Scott Minecrafcie PPE-DE Palmiak Pirkera Stallarholmen Winfryd-Bonifacy Yeosol [ ] ajtemika bio-obrazowania celekoksybem efawirenzem krio-elektronową lajwika non-profit nukleozydowymi przeciwciałem przeciwretrowirusowego ry(d)zykować sakwinawiru tektilandia trichlorobenzenu tuńczykowymi zięciowskim – — ” „
    \item \textbf{memoir lemmas:} ! ) Asińdźka Bludniki Będowszczyazna Bęklowizna Cobary Czołhany Dochorów Dorchów Dłużanin Głowecki Kmińszczyzna Kurypów Lesniowice Muczynowska Nawarya Notiak Pierściorowski Pokasowce Ronantowizna Ruszkowizna Rypnin Rzotoławski Semiginów Siemginów Siemiginów Siemignów Strużewo Stryiskie Swieżaska Szołayska Temerowice Treterówna Treterówną Zebold abbum adlinencja assekuracya całorolny cwansiger daruju domnikalny dotacya dośmierć excentryczność generacya gymnazyum instantacya ioyciec jurysdykcya juryzdyksya kadectwo mandatariat mandataryusz mandatyrusz mojomu mortyfikować mychayłowu niepomiąć nieprzynieść obeymować obeyście ordynarya oycowski pełnłnomocnik przystoyna półgrunt półrolny rarachować separacya spaśne stayermarka submittować sukcessor sukcessorka sukcessya sukcesya szambelanic szyzmatycki treterianum ukochanomu warżyć wdokument świętej pamięci Żółtowizna
    \item \textbf{memoir tokens:}  ! ( ) , Abbum Adelunia Asińdźka Badenianką Bełszowcu Bełzkiem Bienkowskey Blizcey Bludnickey Bludnikach Bludnikami Bludniki Bodzowcu Borkoscy Bołszowcu Bołszowice Bołszowickim Będowszczyazna Bęklowizna Chyrowskiey Cobary Czołhanach Czołhany Dobrrzyńskiej Dochorowie Dominikalnym Domnikalnego Dorchów Dołputowie Dołputów Dziduszyckiego Dłużanie Dźurkowie Floyrana Galecyi Galicyiskiego Golejowskiemi Golejowskimi Gwoźdzu Głoweckiego Głuską Helnkę Horodzyńskiego Inżyniryi Jabłonoscy Jenerałówną Jędrżejowicz Kazimierzostwie Kleofasę Kmińszczyzna Knihinicz Knihiniczach Komornikostwa Komornikowej Kopestyńskich Koropacza Korytyńską Koziobrodzkiego Kołmyiskim Kołomyiskimc Kruszelnicy Kruszelnicę Krzywczas Kunaszowa Kunaszowie Kurypów Kutyszcza Kutyszczach Leboskich Lesniowic Luboscy Maksymowic Mandatariaty Mandataryusz Mandataryusza Mandatyrusza Michałoskich Mohorocie Muczynowską Mychayłowu Nawaryi Naybliższe Nieograniczały Niezabitoski Notiak Obertyńskim Ocyciec Ostaszeskiey Ostaszewskigo Oyciciec Oycowskiey Perekozach Pierściorowskim Pieścioroski Pieściorowski Pokasowce Poldzię Przebysławscy Przebysławski Przebysławskimi Puszczanki Puzyniance Rafałoska Rafałoskiej Romaszówki Ronantowizna Rudźwianach Rusoccy Ruszczewskich Ruszkowizna Rypnin Rzotoławskim Sadogurze Semiginów Separacya Siemginowa Siemiginowa Siemiginowie Siemiginów Siemignowa Simiginowskiego Sopohów Stamisaw Stojoskiego Strużewo Stryiskim Sukcesya Swieżaski Swieżaskiey Swojey Szamanowskim Szołayska Szołayskiego Tarmowiecki Tarnoscy Temerowiec Tomaszowiec Treterianum Treterowej Treterówną Wincentowey Woyniłowa Woyniłowie Woyniłowskich Woyniłów Wołczyńcu Zabilską Załęskiey Zebold adlinencjami administracyę ambicya arędujący assekuracyi asystencyą austyackie balożowaniu bronzowemi całorolni ciaśnieysze ciemnoblnd ciepłey cwansigera cywilney daruju delkatnieyszey domurowanego dotacyi doyrzałemi drugey drżew dway dwukoleśnym dystynkcy ekwipazach etykietalney excentryczności exkluzyi familiinemi foryszyc generacya generacyi gołey gymnazyum główney iOycu installacyę instantacyi jadałney jednająca jedwabney jendory jurysdykcyi juryzdyksye kadectwo kląby kochającey kollokacyi kompano kompleksya kompleksyi koniaż krzewowe kunaszowskie kupsze mandataryusza miarkmi mieycus mieyscowych mojomu mortyfikował najzabawney natarczywiey nawykręcawszy naybliższych nayjstarszego nayjświętszey naymował naymłodszą naymłodzy naypięknieyszey nayskładniey naystarsza naystarszych nayszczęśliwsze naywiększey nayznacznieyszą nibieszczeją niechwytało nieciosanemi niemiano nienosiło nieodjechał niepomięła nieprzyniósł niepsuło nierobiły niespokóy niespotkałem nieusuwały niewinem niezapalają niezostawili nieśniło normalney obedwie obeymował obeyrzeć obeyściu obliwamy obliwać obruciły oczem odchuchano okoloney ordynaryi osypami owalney ożewionemi perswazye pełnłnomocnik piurami piętnastulaty podedworem podeyrzenie policyu pomarłemi porozpuszczały poselskiey pospolitey powiena powiązanemi pruchniało prużnował przedewsią przepiurek przerodni przypiórki przystoyna próżnemi psipiólki póyść półgruntów półrolni rantuchem rarachować redukcyi roumianey ruwieśnika skończoney skrzętnem spaśnemi stancyę starszey stayermarka submittować sukcessorka sukcessorów sukcessyi swojey szambelanica szczupleyszey szczupłey szkrofuliczny szutrowany szyzmatycki sądowey teraźnieysze traktryerni troygiem ukochanomu uprzężone uprzężonym urzędowey warżenia wekslarek wojażerowi wsąsiedztwo wuyem wypiętnowaną wywruż wzorowem zachorowałł zaprzęgył zatabaczone zatrętwiał zawerbował zaymie zaymował zuchwałey zżadka Łomnickiey Łosiówną ładąn Świeżyńską śmieszney śniadey świcząc Żółtowizna – ” „
\end{itemize}
\\
