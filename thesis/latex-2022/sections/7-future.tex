\section{Future Work}
\label{sec:futurework}

A number of the issues mentioned in \autoref{sec:critiques} could be better addressed had the scope of the project been wider and had it been possible to allot more time to it - and it is some of these discarded ideas that form the basis of the potential future work. 

As far as the memoir itself is concerned, completing the annotation thereof with lemmas, UPOS, and XPOS tags would have constituted a small but valuable contribution to the body of annotated historical Polish. It would also have been interesting to see this data with full UD-style annotation, including dependency relations. Furthermore, the library where the manuscript is held appears to be in possession of some of the correspondence by the same author, which could be similarly transcribed and annotated; the memoir's digital version could also benefit from being compared to the contents of the manuscript to eliminate potential transcription errors.

Including more of the data and a fuller annotation could potentially reveal more kinds of variation that may not be evenly distributed within the text. The presence of dependency relation annotation would enable the use of the methods implemented by \citet{johannsen-etal-2015-cross} for a higher-quality analysis of the syntax of the memoir. Refining the methods for utilizing n-gram counts, especially when it comes to the XPOS tags, could yield new insights as well.

Simultaneously, such inquiries pertaining to language variation could be conducted on more data. Both older and more contemporary non-standard data, as well as, potentially, data contemporaneous to the memoir could be explored, and, perhaps, some trends could be identified. Hearkening back to the idea of including more of the author's writing, and referring back to the Korba corpus, which features 17\textsuperscript{th}- and 18\textsuperscript{th}-century texts, the construction and annotation of a diachronic corpus of Polish for a different time period or spanning a larger time period, with the use of a tagset compatible with the UD XPOS tags (if such annotation were to be included) could be extremely beneficial for quantitative investigations into the history of Polish. Alternatively, the focus could be put on regional variation (or on both historical and regional one), as the text discussed in this thesis does display features characteristic of a group of regional dialects.

The experiments reveal certain issues that the tools that were tested struggled with when faced with nonstandard data. While it was not the goal of this project, it could be useful to analyze these issues and explore pre-processing or normalization methods that could be implemented if such tools were to be used for the automatic annotation of larger amounts of historical data, which could be incredibly helpful given the scarcity thereof. 

One direction in which the tagger testing could be developed could be to only review certain tokens where the tagger confidence was below some threshold. Such a method could also be utilized with no golden standard available (on unannotated data). Unfortunately, this idea could not be applied to all of the taggers utilized in this project, as not all of them return confidence scores, at least not in an obvious way.